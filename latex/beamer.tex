%%%%%%%%%%% used for print %%%%%%%%%%%
% \documentclass[handout]{beamer}
% \usepackage{pgfpages}
% \pgfpagesuselayout{4 on 1}[a4paper, border shrink=5mm, landscape]

\documentclass{beamer}

\usetheme{Frankfurt}            % see The Beamer Class for more themes
\usecolortheme{seahorse}
\usecolortheme{rose}
\usefonttheme[onlylarge]{structuresmallcapsserif} % change title font
\usefonttheme[onlysmall]{structurebold}           % change navigation bar font
\setbeamerfont{title}{shape=\itshape,family=\rmfamily}
\setbeamercolor{title}{fg=red!80!black,bg=red!20!white}

\title{Beamer Concepts}
\subtitle{Using Beamer}
\author{fengqi}
\institute{No institute}
\date{\today}


\begin{document}

\begin{frame}
  \titlepage
\end{frame}
% \frame{\titlepage}

\begin{frame}
  \label{outline}
  \frametitle{Outline}
  % \tableofcontents
  \tableofcontents[pausesections]
\end{frame}

\section{Environments and Structure}

\subsection{Environments}
\begin{frame}
  \frametitle{Environments}
  \begin{description}[aligh=left]
  \item [document] {Contains several sections.}
  \item [section] {Contains several frames.}
  \item [frame] {Contains several slides. \\
      Everything between \texttt{\textbackslash begin\{frame\}} and
      \texttt{\textbackslash end\{frame\}} is put on a single slide.}
  \item [overlay(slide)] {Content goes here.}
  \end{description}
\end{frame}

\subsection{Structure}
\begin{frame}
  \frametitle{Structure}
  \noindent
  \[ document
    \begin{cases}
      frame \\
      section
      \begin{cases}
        frame
        \begin{cases}
          overlay \\
          overlay \\
          overlay \\
          \dots
        \end{cases} \\
        frame \\
        frame \\
        \dots
      \end{cases} \\
      section \\
      section \\
      \dots
    \end{cases}
  \]
\end{frame}

\section{Definition}
\begin{frame}
  \frametitle{Definition}
  \begin{definition}
    A \alert{prime number} is a number has exactly two dividors.
  \end{definition}
\end{frame}

\section{Overlays}
\subsection{Example with Overlays}
\begin{frame}
  \frametitle{Example with Overlays}
  \begin{example}
    \begin{itemize}
    \item 2 is prime (two divisors: 1 and 2).
    \item 3 is prime (two divisors: 1 and 3).
      \pause
    \item 4 is not prime (\alert{three} divisors: 1, 2, and 4).
    \end{itemize}
  \end{example}
\end{frame}

\subsection{Overlay Specifications.item}
\begin{frame}[t]
  \frametitle{Specifications.item}
  \framesubtitle{The proof uses \texttt{reductio ad absurdum}.}

  \begin{theorem}There is no largest prime number.
  \end{theorem}
  \begin{proof}
    \begin{enumerate}
    \item<1-> Suppose $p$ were the largest prime number.
    \item<2-> Let $q$ be the product of the first $p$ numbers.
    \item<3-> Then $q + 1$ is not divisible by any of them.
    \item<1-> But $q + 1$ is greater than $1$, thus divisible by some
      primenumber not in the first $p$ numbers.\qedhere
    \end{enumerate}
  \end{proof}
  \uncover<4->{The proof used \textit{reductio ad absurdum}.}
\end{frame}

\subsection{Overlay Specifications.only}
\begin{frame}[t]
  \frametitle{Specifications.only}
  \framesubtitle{The proof uses \texttt{reductio ad absurdum}.}

  \begin{theorem}There is no largest prime number.
  \end{theorem}
  \begin{proof}
    \only<1-> Suppose $p$ were the largest prime number.
    \only<2-> Let $q$ be the product of the first $p$ numbers.
    \only<3-> Then $q + 1$ is not divisible by any of them.
    \only<1-> But $q + 1$ is greater than $1$, thus divisible by some
      primenumber not in the first $p$ numbers.\qedhere
  \end{proof}
  \uncover<4->{The proof used \textit{reductio ad absurdum}.}
\end{frame}

\subsection{Overlay Specifications explained}
\begin{frame}
  \frametitle{Overlay Specifications explained}
  The overlay specifications are given in pointed brackets. The specification
  \texttt{<1->} means ``from slide 1 on.''

  More generally, overlay specification are lists of numbers or numberranges
  where the start or ending of a range can be left open. For example
  \texttt{-3,5-6,8-} means ``on all slides,except for slides 4 and 7.''
\end{frame}

\section{Block}
\begin{frame}
  \frametitle{Block Types}
  \begin{itemize}
  \item \texttt{block}
  \item \texttt{alertblock}
  \item \texttt{definition}
  \item \texttt{example}
  \item \texttt{theorem}
  \item \texttt{\dots}
  \end{itemize}
\end{frame}

\begin{frame}
  \frametitle{What’s Still To Do?}
  \begin{block}{Answered Questions}
    How many primes are there?
  \end{block}
  \begin{alertblock}{Open Questions}
    Is every even number the sum of two primes?
  \end{alertblock}
\end{frame}

\begin{frame}
  \frametitle{What’s Still To Do?}
  \begin{itemize}
  \item Answered Questions
    \begin{itemize}
    \item How many primes are there?
    \end{itemize}
  \item Open Questions
    \begin{itemize}
    \item Is every even number the sum of two primes?
    \end{itemize}
  \end{itemize}
\end{frame}

\section{Columns}
\begin{frame}
  \frametitle{What’s Still To Do?}
  \begin{columns}
    \column{.5\textwidth}
    \begin{block}{Answered Questions}
      How many primes are there?
    \end{block}
    \column{.5\textwidth}
    \begin{block}{Open Questions}
      Is every even number the sum of two primes?
    \end{block}
  \end{columns}
\end{frame}

\section{Verbatim and Code}
\begin{frame}[fragile]          % to use verbatim, fragile is a must.
  \frametitle{An Algorithm For Finding Prime Numbers.}
\begin{verbatim}
int main (void)
{
  std::vector<bool> is_prime (100, true);
  for (int i = 2; i < 100; i++)
    if (is_prime[i])
      {
        std::cout << i << " ";
        for (int j = i; j < 100;
             is_prime [j] = false, j+=i);
      }
  return 0;
}
\end{verbatim}
  \begin{uncoverenv}<2>
    Note the use of \verb|std::|.
  \end{uncoverenv}
\end{frame}

\section{semiverbatim}
\begin{frame}[fragile]
  \frametitle{An Algorithm For Finding Primes Numbers.}
\begin{semiverbatim}
\uncover<1->{\alert<0>{int main (void)}}
\uncover<1->{\alert<0>{\{}}
\uncover<1->{\alert<1>{  \alert<4>{std::}vector<bool> is_prime (100, true);}}
\uncover<1->{\alert<1>{  for (int i = 2; i < 100; i++)}}
\uncover<2->{\alert<2>{    if (is_prime[i])}}
\uncover<2->{\alert<0>{      \{}}
\uncover<3->{\alert<3>{        \alert<4>{std::}cout << i << " ";}}
\uncover<3->{\alert<3>{        for (int j = i; j < 100;}}
\uncover<3->{\alert<3>{             is_prime [j] = false, j+=i);}}
\uncover<2->{\alert<0>{      \}}}
\uncover<1->{\alert<0>{  return 0;}}
\uncover<1->{\alert<0>{\}}}
\end{semiverbatim}
  \visible<4->{Note the use of \alert{\texttt{std::}}.}
\end{frame}

\section{Pictures}
\begin{frame}
  \frametitle{Pictures}
  \begin{figure}
    \includegraphics[scale=0.5]{lion}
    \caption{lion!!}
  \end{figure}
\end{frame}

\section{Hyperlinks and Buttons}
\begin{frame}
  \frametitle{Hyperlinks and Buttons}
  \begin{itemize}
  \item \hyperlink{outline}{Link to Outline}
  \item \hyperlink{outline}{\beamerbutton{Button to Outline}}
  \item \hyperlink{outline}{\beamerskipbutton{SkipButton to Outline}}
  \item \hyperlink{outline}{\beamerreturnbutton{ReturnButton to Outline}}
  \item \hyperlink{outline}{\beamergotobutton{GotoButton to Outline}}
  \end{itemize}
\end{frame}

%%%%%%%%%%%%%%%%%%%%%%%%%%%%%%%
\begin{frame}
  \frametitle{References}

  \begin{thebibliography}{9}
  \bibitem{lamport94}
    Joseph Wright,
    \textit{The BEAMER class}, User Guide for version 3.55,
    Chapter 3: Tutorial: Euclid's Presentation,
    2017.
  \end{thebibliography}
\end{frame}


\end{document}
